\documentclass{article}
\usepackage[left=3cm,right=3cm,top=0cm,bottom=2cm]{geometry} % page settings
\usepackage{amsmath} % provides many mathematical environments & tools
\usepackage{amssymb} % see https://www.overleaf.com/learn/latex/Mathematical_fonts
\newtheorem{thm}{Theorem}  % see http://web.mit.edu/rsi/www/pdfs/theorems.pdf

\usepackage{amsthm}
\theoremstyle{definition}
\newtheorem{definition}{Definition}[section]

\begin{document}

\title{Noncommutative rings}
\author{Herstein}
\date{\today}
\maketitle

\section{Preface}

\subsection{Structures of algebra}

\begin{itemize}
  \item groups
  \item rings
  \item fields
  \item vector spaces
  \item Homomorphisms
  \item Quotient structures
\end{itemize}

\subsection{References}
A survey of modern algebra ; Birkhoff and MacLane \newline
Topics in algebra ; Herstein \newline
Modern algebra ; van der Waerden (the chapter on field theory)

\section{The Jacobson}

\subsection{Modules}

\textit{R} module : 

Module over a ring \textit{R} 

vector space over a ring \textit{R}

\begin{definition}
  The additive abelian group M is said to be an R-module if there is a mapping from $M\times R$ to M (sending (m,r) to mr)
  \begin{enumerate}
    \item m(a+b)=ma + mb
    \item $(m_1 + m_2)a = m_1 a + m_2 a$
    \item (ma)b = m(ab)
  \end{enumerate}
  for all $m \in M$ and all $a, b \in R$.
\end{definition}

If R should have a unit element, \textbf{1}, and if $m\boldmath{1}=m$ for all $m\in M$ we then describe M to be a \textit{unitary} R-module. 

Note that the definition made above merely says that the ring elements induce endomorphisms on M considered merely as an additive abelian group an that furthermore these endomorphisms induced behave as they  should with respect to the addition and multiplication of such endomorphisms. 






















\begin{thm}
  Let X be a complex Banach space and suppose that $T\in \mathcal{B}(X)$ is power bounded. Then 
  \begin{equation} \label{eq1}
    \lim_{n \to \infty} \left\Vert T^n (I-T) \right\Vert =0
  \end{equation}
  if and only if $\sigma(T) \cup \mathbb{T} \subset \{1\}. $
\end{thm}

Here $\mathcal{B}$ deontes the algebra of bounded linear operators on a complex Banach space X, $\sigma(T)$ deontes the spectrum of the operator $T \in \mathcal{B}(X)$ , and an operator $T \in \mathcal{B}(X)$ is said to be power-bounded if $\sup_{n\geq 0}< \infty$. 
Moreover, $\mathbb{T}$ stands for the unit circle $\{\lambda \in \mathbb{C} : |\lambda | =1 \}$.

Limits of the type appearing in \ref{eq1} play an important role for instance in the theory of iterative methods (see 16), so it is natural to ask at what \textit{speed} convergence takes place. 
If $\sigma(T)\cup\mathbb{T}= \emptyset$ the decay is at least exponential, with the rate determined by the spectral radius of $T$, so the real interest is in the non-trivial case where $\sigma(T)\cap\mathbb{T}= \{1\}$.


Given a continuous non-increasing function 
$m : (0,\pi] \to [1,\infty )$ 
such that $\left\Vert R(\mathrm{e}^{i\theta}, T) \right\Vert \leq m(|\theta\|)$ 
for $0 < |\theta| \leq \pi$  , 
it is shown in 21 , Theorem 2.11, that , 
for any $c \in (0,1)$, 
$\left\Vert (T^n (I-T)) \right\Vert = O(m_\mathrm{log}^{-1}(cn)), n \to \infty $
where $m_\mathrm{log}^{-1}$ is the inverse function of the map $m_{\mathrm{log}}$ defined by 
\begin{equation}
  m_{\mathrm{log}}(\epsilon) = m(\epsilon) \mathrm{log} \left(1+ \frac{m(\epsilon)}{\epsilon} \right) , 0<\epsilon \leq \pi , 
\end{equation}






$AX=\lambda X$

$v_1 \begin{bmatrix}
  3\\3\\6\\2
\end{bmatrix}
+
v_2 \begin{bmatrix}
  1\\1\\1\\2
\end{bmatrix}
= 0
$







\subsection*{Lab activity 1.2.4}
Find the difference quotient of $f(x)$ when $f(x)=x^3$.

We proceed as demonstrated in the lab manual; assuming that $h\ne 0$ 
we have
\begin{align*}
    \frac{f(x+h)-f(x)}{h} & =  \frac{(x+h)^3-x^3}{h}   \\
                          & =  \frac{x^3+3x^2h+3xh^2+h^3 - x^3}{h}\\
                          & =  \frac{3x^2h+2xh^2+h^3}{h}\\
                          & =  \frac{h(3x^2+2xh+h^2)}{h}\\
                          & =  3x^2+2xh+h^2
\end{align*} 

\subsection*{Lab activity 2.3.4}
Use the definition of the derivative to find $f'(x)$ when $f(x)=x^{\frac{1}{4}}$.

Using the definition of the derivative, we have
\begin{align*}
            f'(x)           &= \lim_{h\rightarrow 0}\frac{(x+h)^{1/4}-x^{1/4}}{h}   \\
                            &=  \lim_{h\rightarrow 0}\frac{(x+h)^{1/4}-x^{1/4}}{h}\cdot \frac{((x+h)^{1/4}+x^{1/4})((x+h)^{1/2}+x^{1/2})}{((x+h)^{1/4}+x^{1/4})((x+h)^{1/2}+x^{1/2})}\\
                            &=  \lim_{h\rightarrow 0}\frac{(x+h)-x}{h((x+h)^{1/4}+x^{1/4})((x+h)^{1/2}+x^{1/2})}    \\  
                            &=  \lim_{h\rightarrow 0}\frac{1}{((x+h)^{1/4}+x^{1/4})((x+h)^{1/2}+x^{1/2})}   \\
                            &= \frac{1}{(x^{1/4}+x^{1/4})(x^{1/2}+x^{1/2})} \\
                            &=  \frac{1}{(2x^{1/4})(2x^{1/2})}  \\
                            &=  \frac{1}{4x^{3/4}}  \\
                            &=  \frac{1}{4}x^{-3/4}
\end{align*}
Note: the key observation here is that
\begin{align*}
    a^4-b^4 &= (a^2-b^2)(a^2+b^2)   \\
        &= (a-b)(a+b)(a^2+b^2), 
\end{align*}
with 
\[
    a = (x+h)^{1/4}, \qquad b = x^{1/4},
\]
which allowed us to rationalize the denominator.

\end{document}

